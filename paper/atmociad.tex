\documentclass[a4paper,12pt]{book}
\usepackage{graphicx}
\usepackage{aas_macros}
\usepackage{url}
\usepackage{pdfpages}
\usepackage{rotating}
%\usepackage{prelim2e} % Pour faire le bas de page
\usepackage{wrapfig}
\usepackage{afterpage}
\usepackage[square,numbers]{natbib}
\usepackage[english]{babel}
\usepackage[utf8]{inputenc}
\usepackage{hyperref}
\usepackage{listings}
\usepackage{nomencl}
\usepackage{adjustbox}
\newcommand{\Prog}{AeroPlanets }
\makenomenclature

%\usepackage[latin1]{inputenc}
%\usepackage{fullpage}
%\usepackage{pgf}
\author{Guillaume Gronoff \\ Cyril Simon\\ Jean Lilensten}

\title{The \textbf{AtMoCiad} database}
\graphicspath{{images/}{../electron/resultat/}{../photon/resultat/}{../proton/resultat/}{../hydrogen/resultat/}}

\begin{document}
\maketitle
\tableofcontents
%\printnomenclature

\nomenclature{BDD}{Old database. The database used in the very old versions of the Trans* codes. Typically used as a reference when actual references are lost.}
\nomenclature{AtMoCiad}{The Atomic and Molecular Cross sections for ionization and airglow database. The database descripted in the present document.}

\section*{Introduction}


\subsection{History of the project}

This database is the result of several decades of research in aeronomy, most of them happening before I even was born. 
As we will see in this little section, some of the sources and their authors have been lost in the past compilations of data, and we wish to aknowledge them here.

The objective of this project was to gather all the available cross-sections of interest for Aeronomy of the Earth and Planet, and to have a ``clean'' database, with the most possible sources, and the most possible information on the uncertainties.
This latest part became a huge motivation for the project, as the uncertainties are often neglected in the databases, and are the most important parameter for the comparison of the different sources, and for the understanding of the differences between the models.
I came to that realization during my PhD, in Grenoble, France, when I was working on creating a code for computing the airglow and ionization in the atmosphere of Venus under the direction of J. Lilensten. I was using the Trans* model, intitially Transsolo for the Earth, then Transmars/Titan for Mars and Titan; all of these models were developped for a specific planet and had their very specific cross-sections. The code was working well, but as long as we were adding planets, we were adding cross-sections, and the code was becoming more and more difficult to maintain.
When I finished my PhD, I did my post-doc at NASA Langley Research center, in Hampton, Virginia. As a fresh French person coming into the us, the paperwork to let me go on-site took months, and I had to work off-site, with fewer resources for the initial project I was hired for. This is when I decided to create the Aeroplanets model, to have a single code for all the planets, and to have a single database for all the cross-sections. This is how the AtMoCiad project was born.

The code was developped fast enough, in 2009 and 2010, and led to several papers in 2011 and 2012. The database was at a good level, but took more than a decade to come to the current level due to funding, time, etc. In the meantime, work was done on other papers \citep{gronoff2020}, other processes (such as cosmic rays, proton precipitations, etc) and led to incremental improvements of the databse.

The full philosophy and approach is explained in the paper accompanying this database; however, the history of the project is not. This is why I wanted to write this introduction, to aknowledge the people who helped me in this project, and to explain the history of the project.

In the following pages, you will see the word BDD appear. This is the French acronym for Database, and is used when I refer to the cross-sections I found in the Trans* models but for which I never figured out exactly where they came from. 
My understanding is that several of these cross-sections were compiled by Lummerzheim for his PhD \citep{Lummerzheim1987}, but I never found the original sources.  Some must have been compiled for the original Transsolo/Transcar paper \citep{Lummerzheim1994}. These papers likely cover all the cross-sections for N2, O2, O, and Ar. Additional work on N2 and CH4 was done for Titan \citep{Lilensten2005} and Mars \cite{Witasse2002, Witasse2003}. Work on doubly ionized cross-sections was also done for Earth by  \citep{Simon2005}. 
It is likely that other cross-sections computations were used in the creation of what we have here, maybe the work of Phelps \cite{phelps1972} was used somewhere, but I have no way to prove it. One of the reasons for that is the modifications done on cross-sections in the original compilation, due to the way the code was working, especially for the electrons. The cross-sections for the excitation was correctly handled, with a threshold, the energies and values. However, for the ionizations, a main cross-section was used and then a branching ratio was given for all the sub-species, with maybe a threshold for their productions. The code was not able to handle Auger electrons and it created some issues when trying to compute the average energy per ion-pair production. This is why the Aeroplanet model was so crucial and led to the \cite{SimonWedlund2011} paper. In addition, the work on understanding the uncertainties was on its way, and led to the 2012 papers \cite{Gronoff2012, Gronoff2012b}.
I was able to get a small amount of funding for the database around 2014, at a time when money was scarce for NASA projects, and I had to spend too much time writing other grants to fully publish the results.  After that, I was able to get more funding for aeronomy projects, which enabled me to hire a colleague - Bradley Hegyi- part-time to clean the database and to add more cross-sections, and to publish a first version of the project.

Now, additional work has been made to correctly add the H/p cross-sections, even if they are not filled with the same level of details as the other cross-sections. The database is now ready for the community, and I hope it will be useful for the aeronomy and chemistry communities.
I wish to thank all my co-authors and colleagues who helped me in this project. It would never have happened without them.
The errors, omissions, and mistakes in this database are all my fault, and I hope the community will help me correct them in the future.


G.Gronoff,
September 2024, Hampton, Virginia, USA









\subsection{Presentation of the database}



Ionization, excitation, dissociations are terms of the utmost importance when dealing with the physics of upper atmospheres.

Several physical processes are at the origin of the creation of an ionosphere for the planets. The same are also at the origin of the heating, and the formation of different kinds of  airglow and aurorae in the thermospheres.

To model these upper atmosphere, of either Earth, planets, or comets, along as other kinds of plasma, some parameters are needed.
Among them, the cross sections for ionization, excitation, dissociation\ldots

Up to now, no centralized database was available for these processes, leading to several problems:
\begin{itemize}
\item Each team has its own database, with its own history. Therefore, some processes are up to date, some very old\ldots
\item It is impossible to compare two team databases, because of their standards. It is also impossible to compare two codes by different teams, since the high dependence on cross sections does not allow to discriminate whether the differences come from the implementation of the physical processes or from the inputs.
\item Each time a cross section is measured, it takes a very long time to diffuse in the community, highlighting the precedent issues.
\item The chemists and experimentalists, and especially the people measuring and/or computing the cross sections, hardly communicate with the aeronomy community. Therefore, they do not know the needs for specific cross section improvements, and have to perform a very important bibliographic work to compare their measurements with previous work. Unfortunately, such efforts are likely to be only partly published, leading to the same efforts for the following teams. %Besides the fact that this work is likely to be lost in the increasing number of papers, the lack of a database also 
\item On the contrary, aeronomists can detect problems in dataset when they compare models and experiments. It is also possible for them to improve estimation of absolute values, and extrapolation of existing datasets. These results, independently of their quality, never reaches the chemistry community.
\end{itemize}

Among these problems, one parameter was always neglected in the previous models and databases: \textbf{the uncertainty of the cross sections}.

This term is one of the most important for the comparison of the different sources, but also to understand the different outputs of the models.

When we compare the sources, the claimed uncertainty the most important parameter for extracting the recommended cross section. The other cross sections can sometimes be used to refine the absolute value (correction of the discrepancy).

For the models, the cross sections are the main sources of uncertainties. Therefore, the uncertainty of the model can be computed to a good extent through the variations of the cross sections. A fast technique is the use of a Monte Carlo technique (it will be used in the following to estimate the quality of the electron cross sections dataset).


All these problems leads to the creation of a centralized database for cross sections, which makes a link between notably the aeronomy and chemistry communities. And has to include  the uncertainties,  the extrapolations and the most possible number of different sources for each processes, in order to select the recommended uncertainty.

This database, described in the document, is named \textbf{AtMoCiad}, which stands for \textbf{At}omic and \textbf{Mo}lecular \textbf{C}ross section for \textbf{i}onization and \textbf{a}irglow\footnote{or \textbf{a}urora} \textbf{D}atabase\footnote{For the connoisseurs, it is also a recursive acronym: \textbf{AtMoC} \textbf{i}s \textbf{a}nother \textbf{D}atabase}.



In a first chapter (\ref{physics}), we will describe the physics associated with these cross sections, and how to use them correctly. That part will briefly describes the \textit{Aeroplanets} model, which is used to check the dataset; but which is also the first aeronomy model using \textbf{AtMoCiad}.

In a second chapter (\ref{photoionization}), we will present the data for the ionization, excitation and dissociation by the photons.

In a third chapter (\ref{electronimpact}), we will present the electron impact cross sections.

In a fourth chapter (\ref{evolution}), we will explain the future improvements of the database (proton impact cross sections\ldots).


In the appendixes (\ref{appendixes}), we will describe the technical storage, plotting and web interfacing of the database.


%\citet{Gronoff2008}



%\include{part1_biblio/saber_retrieval}

\include{physics}

\chapter{Photoionization}
\label{photoionization}

\section*{Introduction}

\nomenclature{AMOP}{Photoionization cross sections database compiled by the AMOP group: \url{http://amop.space.swri.edu/}. }




\input{../photon/resultat/CO2}
\input{../photon/resultat/O}
\input{../photon/resultat/O2}
\input{../photon/resultat/CO}
\input{../photon/resultat/N2}
\input{../photon/resultat/CH4}
\input{../photon/resultat/Al}
\input{../photon/resultat/Ca}
\input{../photon/resultat/C}
\input{../photon/resultat/Fe}
\input{../photon/resultat/H2O}
\input{../photon/resultat/H2}
\input{../photon/resultat/He}
\input{../photon/resultat/HO}
\input{../photon/resultat/H}
\input{../photon/resultat/K}
\input{../photon/resultat/Mg}
\input{../photon/resultat/Na}
\input{../photon/resultat/NO}
\input{../photon/resultat/N}
\input{../photon/resultat/Si}
\input{../photon/resultat/S}
\input{../photon/resultat/Ti}


\chapter{Electron impact}
\section*{Introduction}
\label{electronimpact}

%\input{../electron/resultat/CO2}
%\input{../electron/resultat/O}
%\input{../electron/resultat/O2}
%\input{../electron/resultat/CH4}
\input{../electron/resultat/Ar.tex}

%\input{../electron/resultat/CO}

\input{../electron/resultat/CH4.tex}
\input{../electron/resultat/CO2.tex}
\input{../electron/resultat/CO.tex}
\input{../electron/resultat/H2O.tex}
\input{../electron/resultat/H2.tex}
\input{../electron/resultat/He.tex}
\input{../electron/resultat/H.tex}
\input{../electron/resultat/N2.tex}
\input{../electron/resultat/O2.tex}
\input{../electron/resultat/O.tex}



\clearpage
\section{Cross section of  impact with  Protons}
\subsection{Inelastic Cross Sections}

%\subsubsection{Legend for the properties}
%\paragraph{ R }: Recommended cross section for the processus. It is used in the main file. The selection of the recommended cross section is based on the quality of the data (e.g. errorbars, comparison with other experiments), the possibility of extrapolation, and the origin of the work, coupled with the consistency (sum of recommended cross sections ~ Total cross section)
%\paragraph{ U }: Estimated uncertainty: sometimes, the uncertainty is not given, because of theoretical work... The authors of the database have to estimate the uncertainty, but the quality of that estimation can be questionable. Moreover, when data from different sources have been adapted (e.g. for extrapolation), the uncertainty can be modified...
%\paragraph{ E }: Validated for extrapolation: the extrapolation of these cross sections is plausible. For example, when an analytic function has been applied...



\input{../proton/resultat/CO2.tex}
\input{../proton/resultat/CO.tex}
\input{../proton/resultat/H2.tex}
\input{../proton/resultat/H.tex}
\input{../proton/resultat/N2.tex}
\input{../proton/resultat/O2.tex}
\input{../proton/resultat/O.tex}

\clearpage

\section{Cross section of  impact with  Hydrogen}


\input{../hydrogen/resultat/CO2.tex}
\input{../hydrogen/resultat/N2.tex}
\input{../hydrogen/resultat/O2.tex}
\input{../hydrogen/resultat/O.tex}


%\chapter{Photoionization}
\label{photoionization}

\section*{Introduction}

\nomenclature{AMOP}{Photoionization cross sections database compiled by the AMOP group: \url{http://amop.space.swri.edu/}. }




\input{../photon/resultat/CO2}
\input{../photon/resultat/O}
\input{../photon/resultat/O2}
\input{../photon/resultat/CO}
\input{../photon/resultat/N2}
\input{../photon/resultat/CH4}
\input{../photon/resultat/Al}
\input{../photon/resultat/Ca}
\input{../photon/resultat/C}
\input{../photon/resultat/Fe}
\input{../photon/resultat/H2O}
\input{../photon/resultat/H2}
\input{../photon/resultat/He}
\input{../photon/resultat/HO}
\input{../photon/resultat/H}
\input{../photon/resultat/K}
\input{../photon/resultat/Mg}
\input{../photon/resultat/Na}
\input{../photon/resultat/NO}
\input{../photon/resultat/N}
\input{../photon/resultat/Si}
\input{../photon/resultat/S}
\input{../photon/resultat/Ti}




%\chapter{Electron impact}
\section*{Introduction}
\label{electronimpact}

%\input{../electron/resultat/CO2}
%\input{../electron/resultat/O}
%\input{../electron/resultat/O2}
%\input{../electron/resultat/CH4}
\input{../electron/resultat/Ar.tex}

%\input{../electron/resultat/CO}

\input{../electron/resultat/CH4.tex}
\input{../electron/resultat/CO2.tex}
\input{../electron/resultat/CO.tex}
\input{../electron/resultat/H2O.tex}
\input{../electron/resultat/H2.tex}
\input{../electron/resultat/He.tex}
\input{../electron/resultat/H.tex}
\input{../electron/resultat/N2.tex}
\input{../electron/resultat/O2.tex}
\input{../electron/resultat/O.tex}


%\include{part3_protons/Protons}

\chapter{Sources for specific state or ions}
\section{Automatic computation of the list}
The list can be automatically generated from the different files, or the recommended cross section file.

For consistency reasons, the following list was generated from the recommended set of data. The main reason lies in the non-recommended files: sometimes, the authors cannot discreminate between two species with their experiment, and then, gives the production for the two species (example: O$_2^{++}$ and O$^+$ cannot be discriminated in standard ion mass spectrometer). It is stated in the specie list for that cross section in the non-recommended set, while  in the recommended one, we tried to extract the species producted).

%\section{Sources and plot reference for specie production}

%\include{finalspecies}






\chapter{Evolution of the database}
\label{evolution}

We have a basic set of cross-sections for the impact of Hydrogen and Protons, however, this part of the database is quite sparse and should be further improved.

Other evolution for the database include taking into account the temperature dependence of the cross-sections, notably for the photodissociation of molecules at cross-sections in the 200-300 nm range. This would require a more detailed study of the temperature dependence of the cross-sections, and the implementation of a temperature-dependent database. This is particularly important for the deeper parts of the atmospheres, such as the O$_3$ layer, where the temperature-dependence of the cross-sections can be significant.

Another potential source of evolution is for the ionization of ions and multiple-ionization cross-sections. 




\chapter{Conclusion}

The \textbf{AtMoCiad} database, presented here, will soon be released on internet, with free access. Contrarily to older databases, the design, presented in the appendixes, was developed for frequent modification and updates. Moreover, it is intended to be extended and completed for the scientific community, by the scientific community.

More than an useful tool for each user, \textbf{AtMoCiad} is also the basis for comparing the different codes developed by the modelers, along with a reference for the experimental community, and especially the laboratory that measure these cross sections.  



\appendix
\label{appendixes}
%\section*{Appendix}
\chapter{The XML files}

\nomenclature{XML}{eXtensible Markup Language: document format based on Markup. To simplify, it is a text (ASCII/UTF8....) document containing markups to define its structure, and the relations between the several parts of the text. }

\nomenclature{XPath}{Set of techniques to easily navigate in a XML document. The basic concept is to follow the 'roots' to the 'leaves' of the document, by giving all the nodes. An example can be found in \ref{Hierarchy}. Thanks to the wide use of XML, notably in web-based technology, XPath techniques are implemented in several programming languages. For example, python has the ElementTree module, extensively used in the plotting and transforming examples. }

\nomenclature{\Prog}{Software developed by G. Gronoff on the basis of the Trans* codes. It is an upper atmosphere of the earth and planet model, for computing ionization, dissociation and excitation. It has several modules for computing the emissions, for retrieval\ldots It is the first model using the \textbf{AtMoCiad} database, and its development was useful for determining the necessary parameters of the database xml files.}

\section{Working with XML files}

XML is a kind of standardized ASCII file. It looks like HTML, but strict (markup should be closed, no bracing like $<a><b></a></b>$ but $<a><b></b></a>$).
It is well suited for defining configuration, especially for communication between different softwares. Unfortunately for scientific purpose, it does not explicitly defines the notion of array (some specific files format, like hdf, are more suited when dealing with very large data files): list of numbers are used as array for the following.

Several advantages can be found in XML files:
\begin{enumerate}
	\item Flexibility: you can make a lot of commentaries: reading XML files does not depend on the position of your data (contrary  to the typical interface file in Fortran, for which the number of value to read has to be stated).
	\item Size of data: you don't have to declare the size of your arrays, it is computed automatically.
	\item Position of your information: you can put the information wherever you want! For example \Prog uses a XPath wrapper to read the file: it depends on the hierarchy and the name of the markup.
	\item Possibility to duplicate markup. Typically the output of selected species production! You can select different species\ldots
	\item Mixing strings and numbers! This allows  REALLY flexible softwares. Typically in this case, you define the species, and its cross section. Therefore, the addition of a new species can be straightforward\footnote{It is especially true in \Prog, you can define a new species in your database, then add its atmospheric model in the configuration file! In other terms, you can add a specie in a planet model without modifying the source code and so without recompilation.}.
	\item Widely used formating system. It is easy to write
\end{enumerate}

Of course, when implementing a full support of XML files, you have to take into account several points:
\begin{itemize}
	\item XML is not a real user-computer interface. But editors exists, and it is the best format if you plan to add a web-based interface.
	\item XML is not dedicated to large dataset. When dealing with very big files, you should consider using HDF (binary, but without endianness problems).
	\item For the present database, the need for flexibility (frequent modifications), for interfacing  (plot, python, C++), and for ease of integration into a system dedicated for several planets, oriented us toward the direct use of XML files\footnote{For people considering using other file format, the best solution is to use python-elementtree to create a conversion software.}
\end{itemize}

As you can see, the authors of this documents highly encourage the use of the XML (and HDF) file format.

Anyway, to have a comprehensive XML file, i.e. really flexible and usable in your model, several concepts needs to be understood.

\subsection{Hierarchy}
\label{Hierarchy}

The hierarchy of the markup is very important. Hierarchy means dependence. It is well suited for large set of options.
Example, to define the mass and the ground state of CO2, the XML way could be:
\begin{verbatim}
<!-- The XML comment are inside this king of strange brackets! -->
<CO2>
	<!-- Mass in amu -->
	<mass>44.00995</mass>

	<!-- List of possible states, beginning with the ground state -->
	<states>
		X
	</states>
	<!-- ... -->
</CO2>
\end{verbatim}

Here, state and mass are hierarchically inferior to CO2. (Note the way we write commentaries $<!--$ $-->$). 
Such writing allows to define several species, with the same template\footnote{Note to developers: XPath defines nodes to work with such templates.} .

\begin{large}\textbf{In the following, the hierarchy for the options will be written in a XPATH-style.}
\end{large}

It means that the ``mass'' option in the precedent example will be presented as /CO2/mass. 


\subsection{Markup keys}

Sometimes\footnote{Strictly speaking, use of keys can be avoided in XML: it could be replaced by the use of a hierarchically inferior markup. Anyway, we sometimes prefer not to write the closing markup: understandable names are sometimes very long\ldots}, we can use a key inside the markup $<truc~key='42' >bla</truc>$.
We use this option to add simple statements, like model type:
\begin{verbatim}
	<!-- altitude grid -->
	<alt_grid>
		<!--
		use_model type:
			0 : standard grid
			1 : data
		-->
		<use_model type="0"/>
		<!--
			If we use a standard grid, the options are:
				type 0 : exp decrease
				type 1 : power law decrease
				type 2 : constant width
		-->
			<st_grid type="0">
				<altmin>120</altmin>
				<altmax>300</altmax>
				<number>50</number>
			</st_grid>

			<altdata></altdata>

	</alt_grid>
\end{verbatim}

This facility is really powerful when we work with uncertainties:
\begin{verbatim}
			<Cross unit="cm2" uncertainty="30%">
			...
\end{verbatim}
We can define the uncertainty for an array really easily. 


\subsection{Numerical values}

Two key are very important for the database, the ``fact'' and ``uncertainty'' keys.

These are used a lot in the definition, and bad interpretation of the database could come from the misuse of these keys.

\subsubsection{The ``fact'' key}
The numerical values can be modified thanks to the ``fact'' key.
For example:
\begin{verbatim}

<a fact='12'>1</a>
\end{verbatim}
will give a result of $12$ if it is treated as a numerical value.
\subsubsection{The ``uncertainty'' key}

As explained before, this key is used to define the uncertainty of the dataset. If used with a \% sign, it is perceived as a percentage. If not, it should be taken as the 1-$\sigma$  difference.


\section{The  cross section files}

The cross section XML files are the core of the database. They can easily be modified, transformed into other kind of files, or automatically plotted with the specific tools.

These standardized files also contains information to automatically create the web and pdf interfaces to the database. The photoionization cross sections files are the simplest, because more information are needed for the electron files. Therefore, we will describe the photon files before the improvements in the electron files.

The recommended cross section files are the dataset that should be used in the different softwares. In \Prog, these XML files are directly used as inputs.


\subsection{The heading of the cross section files}


To ensure the validity of the documents, it start and ends with the ``$<$crs$>$/$<$/crs$>$'' markup. 
The cross sections, should be put in the valid format inside a SPECIE markup. For example, for a file with CO$_2$ and O$^+$ cross sections, the file should look like:

\begin{verbatim}
<crs>
	...
	<CO2>
		...
	</CO2>
	...

	<O_PLUS>
		...
	</O_PLUS>
</crs>
\end{verbatim}

Because the `+'  character is not allowed inside the XML markup name, it has to be transformed into \_PLUS in the database.

Strictly speaking, the cross section files are meant to describe only one species, and only one kind of interaction. 


This allows an automatic plot and some other automatic interactions, by putting some markup specific to that file.

For example, if one file contains the cross sections for one process, from different sources. We could define a specific title, limits for an automatic plot, plotname...


\paragraph{/crs/Name :} The name of the species that reacts. (*)\footnote{(*) stands for mandatory options.}
\paragraph{/crs/Collider :} The name of the particle that impact with the species (ph for photons, e for electrons, and soon p for protons). (*)
\paragraph{/crs/RecommendedFile :} If the RecommendedFile markup is present, the system knowns that it is likely to be the whole set for the different cross sections. It is notably used inside the plotting system to plot the name of the process for each cross section, and not the extended legend.
\paragraph{/crs/title :} Title of the cross section file, typically written in latex: the pylab library is able to transform formulae for the title of the plots. (*)
\paragraph{/crs/Emin :} For plotting: lower energy boundary. (*)
\paragraph{/crs/Emax :} For plotting: upper energy boundary. (*)
\paragraph{/crs/Cmin :} For plotting: lower cross section boundary. (*)
\paragraph{/crs/Cmax :} For plotting: upper cross section boundary. (*)
\paragraph{/crs/plotname :} The name of the standard plot (no extrapolation, abscissa unit is energy in eV). (*)
\paragraph{/crs/lambplotname :} If present, creates a plot against wavelength (no extrapolation, abscissa unit is in angstrom).
\paragraph{/crs/explotname :} If present, extrapolates the standard plot (extrapolation, abscissa unit is energy in eV)
\paragraph{/crs/exlambplotname :} If present, extrapolates the wavelength plot (extrapolation, abscissa unit is in angstrom).




\paragraph{Example: the header for the recommended CO$_2$ + $\lambda$ cross section file}


\begin{verbatim}
<crs>
	<Name>CO2</Name>
	<Collider>ph</Collider>
	<RecommendedFile/>
	<title> CO$_2$  + $\lambda$</title>
	<Emin>5</Emin>
	<Emax>1E5</Emax>
	<Cmax>1E-15</Cmax>
	<Cmin>1E-28</Cmin>
	<plotname>seff_CO2_ph_recommended.pdf</plotname>
	<lambplotname>seff_CO2_ph_recommended_lambda.pdf</lambplotname>
	<explotname>seff_CO2_ph_recommendedex.pdf</explotname>
	<exlambplotname>seff_CO2_ph_recommended_lambdaex.pdf</exlambplotname>
	<CO2>
		...
	</CO2>
</crs>
\end{verbatim}



\subsection{The photoionization cross section file (cross section basic file)}

The cross sections works like the species file concerning the name (except that the highest markup\footnote{All the files will have an highest markup now: it allows to look at the XML file with firefox, and therefore to allows firefox to detect the XML errors!} is ``crs''.

\paragraph{ /crs/species}
The species markup. Highest markup for the real definition.

\paragraph{/crs/species/TotalCrs}
\label{cross_sec_photoioni_total}

This markup allows to define the total cross section. Very useful when we know it: it is really more precise than sum up ionization cross section.

The typical use of this markup is : 
\begin{verbatim}
		<TotalCrs>
			<Egrid unit="eV">
			</Egrid>
			<Cross unit="cm2">
			</Cross>
		</TotalCrs>
\end{verbatim}

Typically, the unit can have the uncertainty key:
\begin{verbatim}
	<Cross unit="cm2"  uncertainty="30%">
	<Cross unit="cm2"  uncertainty="O.1">
\end{verbatim}

You can define an 1$\sigma$ uncertainty in percentage: very useful when your cross section varies on a huge range of parameters!
But you can also give a relative uncertainty, in the example, 0.1 stands for $\pm0.1$.

In both sub-markup, a ``unit'' key is defined. It is not used in \Prog now\footnote{Boost has a unit module\ldots}, but:
\begin{itemize}
	\item It is better when you check your data.
	\item It allows to be used of the standard units for the code.
	\item It can really be useful when used with other sources.
\end{itemize}

Now, the only accepted units are ``eV'' for the energy grid and ``cm2'' for the cross sections.


Of course, you could use the ``fact'' option to modify the  global value, this is the best way to transform Barn into cm$^2$\ldots

% WARNING: if you do MonteCarlo simulation, the total cross section is not checked against the sum of the other cross sections! The reason is simple: when the total cross section is defined, the other cross sections are regarded as a factor for the computed flux. Therefore, you can use several cross section for the processes.
%Example : process $CO_2+h\nu\rightarrow CO^+ + O^*$ can have a subprocess $CO_2+h\nu\rightarrow CO^+ + O(^1S)$.

%\paragraph{/crs/species/ZeroCrs}
%If you do not have a cross section for the species, you can define ZeroCrs.

\paragraph{/crs/species/TotalCrsIsTheSum}
As stated in the main document, for photoionization, it is better to have a total absorption cross section, instead of adding the other cross sections, because the uncertainty of the flux would be decoupled from the uncertainties of each processes which can be variable.
Anyway, if you do not have a total cross section, just add
\begin{verbatim}
<TotalCrsIsTheSum/>
\end{verbatim}
it allows to define that the total cross section is the sum of the other.

%\paragraph{/crs/species/DisableTotalCrsWarning}
%In some very special cases, no total crs is computed or defined. This must be used.

\paragraph{/crs/species/Process}
keys: name of the process, number of electrons, and threshold.
It is also possible to add ions, very useful when double ionization. Note that the number of ions of electrons are floating point values\footnote{Useful when the produced species are determined with a branching ratio.}!

The number of electrons could be 0 even if there is an ionization. This is a technique used to take into account one excited state of an ion while the total ionization is computed through a more precise cross section.


\begin{verbatim}
		<Process name="" electrons="" threshold="">
		<!-- also possible:
		<Process name="" electrons="" threshold="" ions="">
		-->
			<Species>
				<Specie name="" state=""/>
				<Specie name="" state="" number=""/>
			</Species>
			<Egrid unit="eV">
			</Egrid>
			<Cross unit="cm2">
			</Cross>
		</Process>
\end{verbatim}

\paragraph{/crs/species/Process/Species}
The Species defines a list of species created through the process.
\paragraph{/crs/species/Process/Species/Specie}
One of the created species (you can have several species, or 0! -but it is not really useful unless this is necessary for total crs sum- ).
The key for the species is its name and its state. The state can be X. It allows to compute the production of excited states!
One of the very important point for the species state is the possibility to add ``-NOTOT'' at the end of the state.
 It defines that the state created for the species has already been counted in the total species production.

 For example, the electron production for CO$_2^+$ can be computed without taking into account the different states. But if we need the A state, we have to add the Itikawa 2002 cross section. If we need both total production (accurate) and A state production, we just need to add -NOTOT at the end of the state name like in this example. 

If we do not have the precision for the state computed, typically for an ion, the best way is to define the excited state as ``Total''. 
\begin{verbatim}
		<!--  Itikawa 2002  for CO2+(A)-->
		<Process name="CO2+e -> CO2+(A) " electrons="0" threshold="17.32">
			<Species>
				<Specie name="CO2+" state="A-NOTOT"/>
			</Species>
			<!-- The ionization is not taken
			into account, because
			it is a subproduct of the total ionization
			-->
		...
		</Process>
\end{verbatim}

On the contrary, some processes defines the production of the X, A, B,\ldots states. But not the total production. In that case, we do not add the -NOTOT at the end, and, by detecting that fact, we know that we must add each processes to get the total production.

The addition of -NOTOT is therefore specific to the recommended file, and depends on the whole set of cross sections.




\subsubsection{Concerning the electron production}
The electron production is computed by adding the electron production of all processes, thanks to the number of electrons parameters. If you want to define a subprocess, you must let the number of electron produced at 0!

\subsubsection{Shirai cross sections}
%The electron cross sections from Shirai et al. (2001,2002 and Tabata,Shirai 2006) can also be included (if one day, this kind of parametrisation of cross section is done for photoionisation, it could be installed also).
The electron cross sections from \citet{Shirai2001,Shirai2002,Tabata2006} can also be included (if one day, this kind of parametrisation of cross section is done for photoionisation, it could be adapted too).

\paragraph{/crs/species/Process/Shirai} To define that we are using the Shirai/Tabata cross sections. An example of the interface for these cross sections files could be seen in the section \ref{autoplottingsystem}.
\paragraph{/crs/species/Process/Emin} The minimum energy where it is defined (For non-Shirai system, it is the energy of the first non-zero data point. It can be automatically determined there. )
\paragraph{/crs/species/Process/Emax} The maximum energy where it is defined (For non-Shirai system, it is the energy of the last non-zero data point. When the system is extrapolated, it could be a point defined for having a good shape, or the point where the extrapolation is expected to be valid It can be automatically determined there.).
.
\paragraph{/crs/species/Process/Equation} Gives the type of the equation, and the article id (and the number in the article, but not used). Since the equations are dependent upon the species and the number, we are not reproducing them here. However, the example codes for plotting the cross-sections show an implementation of the equations. 

\paragraph{/crs/species/Process/params} Gives the parameters for the cross section.


\begin{verbatim}
<Process name=``CO2+e -> O + C'' electrons=``O'' threshold=``11.100000''>
		<Shirai/><!--Shirai et al 2001 analytic cross section -->
	<Species>
		<Specie name=``C'' state=``X''/>
		<Specie name=``O'' state=``X'' number=``2''/>
	</Species>
	<Emin> 13.5 </Emin>
	<Emax> 199.0 </Emax>
	<Equation type=``1'' article_id=``CO2'' article_number=``36''/>
	<params>
		7.040000e-01	  1.084000e+00	  2.680000e-02	  5.700000e-01
	</params>
</Process>
\end{verbatim}


The equation types for the Shirai cross sections are:

\subsubsection{Singhal cross sections}
The electron cross sections from \citet{Singhal2009} can also be included. The interface is the same as the Shirai cross sections, but the markup is different.

\paragraph{/crs/species/Process/Singhal} To define that we are using the Singhal cross sections. 
\paragraph{/crs/species/Process/params} Gives the parameters for the cross section.
\paragraph{/crs/species/Process/AF} Gives the autoionization factor for the process.
\paragraph{/crs/species/Process/Omega} Set up if the equation for this cross section is of Omega type
\paragraph{/crs/species/Process/Ctype} Set up if the equation for this cross section is of C type

\paragraph{Implementation of the cross-sections}
There are 3 types of cross-sections for Shirai: first the ionization cross-sections type, then the excitation cross-sections types C and Omega. 
Once you get the parameters from the database, from parray[0] to parray[n], and the energy E for the computation, you can go as follows:

parray[0] is the threshold, so any E lower than parray[0] will return 0.0.
We define
\begin{equation}
    q0=6.513E-14
\end{equation}
This allows to return a cross-section in cm$^2$.

For a Omega type cross-section, the cross-section is computed as:
\begin{eqnarray}
	ratio &=& \frac{parray[0]}{E} \\
	Term1 &=& \frac{q0 \times parray[5]}{parray[0]^2} \\
	Term2 &=& (1 - (ratio ^ {parray[1]})) ^{parray[2]}\\
	Term3 &=& ratio^{parray[4]}\\
	CrossSection &=& Term1 \times Term2 \times Term3
\end{eqnarray}
For a C type cross-section, the cross-section is computed as (log is the natural logarithm):
\begin{eqnarray}
	ratio &=& \frac{parray[0]}{E} \\
	Term1 &=& \frac{q0 \times parray[5]}{E \times parray[0])} \\
	Term2 &=& (1 - (ratio ^ {parray[1]})) ^{parray[2]}\\
	Term3 &=& \log( e + 4 * parray[4] / ratio)\\
	CrossSection &=& Term1 \times Term2 \times Term3
\end{eqnarray}
For the ionization cross-section, the cross-section is computed as:
%    Sigma_0 = 1E-16
%            A_E = (parray[1] / (E + parray[2])) * log((E / parray[3]) + parray[4]+(parray[5] / E))
%            Gamma = (parray[6] * E) / (E + parray[7])
%            T_0 = parray[8] - (parray[9] / (E + parray[10]))
%            T_m = 0.5*(E - parray[0])
%            Term1 = A_E
%            Term2 = Gamma
%            Term3 = arctan((T_m-T_0) / Gamma) + arctan(T_0 / Gamma)
%            Cross = Sigma_0 * Term1 * Term2 * Term3
\begin{eqnarray}
    \sigma_0 &=& 1E-16 \\
    A_E &=& \frac{parray[1]}{E + parray[2]} \times \log\left(\frac{E}{parray[3]} + parray[4] + \frac{parray[5]}{E}\right)\\
    \Gamma &=& \frac{parray[6] \times E}{E + parray[7]}\\
    T_0 &=& parray[8] - \frac{parray[9]}{E + parray[10]}\\
    T_m &=& 0.5 \times (E - parray[0])\\
    Term1 &=& A_E\\
    Term2 &=& \Gamma\\
    Term3 &=& \arctan\left(\frac{T_m-T_0}{\Gamma}\right) + \arctan\left(\frac{T_0}{\Gamma}\right)\\
    CrossSection &=& \sigma_0 \times Term1 \times Term2 \times Term3    
\end{eqnarray}

    




\paragraph{Example of implementation in the database}
\begin{verbatim}
	<Process electrons="0" name="H2O+e -> H2O(A2v2)" threshold="0.391">
	    <Species>
		<Specie name="H2O" state="A2v2"/>
	    </Species>
	    <Legend>H2O(A2v2)</Legend>
	    <Proc>H2O(A2v2)</Proc>
	    <Section>excitation</Section>
	    <Source type="review">Singhal</Source>
	    <Notes>from the book Elements of Space Physics</Notes>
	    <uncertainty>30%</uncertainty>
	    <EstimatedUncertainty/>
	    <Excitation/>
	    <Singhal/>
	    <params>0.391 1.000 3.000 0.391 6.000 0.000043 0.0</params>
	    <AF>0.0</AF>
	    <Omega/>
    	</Process>
	
		<Process electrons="0" name="H2+e -> C1Pu" threshold="12.465">
			<Species>
				<Specie name="H2" state="C1Pu"/>
			</Species>
			<Legend>C1Pu</Legend>
			<Proc>H2(C1Pu)</Proc>
			<Section>excitation</Section>
			<Source type="review">Singhal</Source>
			<Notes>from the book Elements of Space Physics</Notes>
			<uncertainty>30%</uncertainty>
			<EstimatedUncertainty/>
			<Excitation/>
			<Singhal/>
			<params>12.465 0.85 1.464 12.465 0.300 0.392300</params>
			<AF>0</AF>
			<Ctype/>
		</Process>
		
\end{verbatim}



\subsection{The Electron cross section file (extends the standard cross section!)}

The electron cross section is basically the same file as the photoionization cross section\footnote{Heritage in C++! A fantastic concept!} .
The main difference is that the total cross section is not used. So, we generally define it as TotalCrsIsTheSum.

The total cross section is replaced by elastic, ionization and excitation cross sections (ionization and excitation are the inelastic cross sections, these cross sections have technically almost no differences, except being computed separately).

\paragraph{/crs/species/ElasticCrs}
Elastic cross section, defined like every other cross sections.
%If NoStdExtrapolate is set, the Log Log interpolation will be used to extrapolate the data. If the markup is not set, the elastic extrapolation is used: the energy decrease is proportional to $E^{-0.65}$ between 400 and 2000 eV, and is proportional to $E^{-1}$ above that energy.



\paragraph{/crs/species/ExcitationCrs, /crs/species/IonizationCrs}
Allows to define total excitation and ionization cross sections.
These options are NOT RECOMMENDED. Because energy conservation for electron impact is computed by using thresholds: energy conservation is not computed in that case!

We recommend to use Excitation and Ionization!

\paragraph{/crs/species/Process/Excitation}
Allows to specify that this is an excitation process that should be use to compute the total excitation cross section. 

\paragraph{/crs/species/Process/Ionization}
Allows to specify that this is an ionization process that should be use to compute the total excitation cross section. 
\subsubsection{Concerning the total electron production}
The total electron production is computed by adding the ionization process results (computed by the number of electrons).
Therefore, if you use a subprocess (of a previously defined process), you should consider that its ionization is already taken into account.
If you defines that this sub-process creates also an electron, if the first process creates an electron, the result will be a double-ionization (probably not what you want in that case).

When more than one electron are created in the process, the electrons production is correctly stored, but for the electron flux, all the energy is put inside one electron.

When there is an Auger process, the Auger electron is correctly put in the flux.

\subsubsection{Auger electron computation}

The Auger electron process is taken into account in \Prog. The principle is simple: a suborbit is ionized (creates one electrons) and one electron of an upper orbits falls to that orbit, creating a photon that is absorbed by the species itself; therefore, creating another ionization.
This leads to a double ionization, with a second electron very specific: it has the energy of the transition; it is an auger electron.

In the \textbf{photoionization} and \textbf{electron impact} cross section, you can add the Auger process simply by defining a cross section, and adding the Auger markup:

\paragraph{/crs/species/Process/Auger}
Allows to define the Auger process, and the Auger electron energy:
\begin{verbatim}
<Auger energy="500"/>
\end{verbatim}


Several Auger electrons can be created in one process, with different efficiency, you can simply do:

\begin{verbatim}
<Auger energy="500" fact="0.5"/>
<Auger energy="800" fact="0.5"/>
\end{verbatim}

If no efficiency is defined, it is considered to be one.


\subsection{The recommended data set cross section}

The recommended data set cross section works technically like the other files.
The main difference is the ``/crs/RecommendedFile'' in its heading.
This markup could be used for discriminating against the other types, for example by displaying the process and 
not the sources when plotting. 

The recommended data set is not automatically created from the other files right now. This may become true in the future, and, in that case, people writing the database should not care about the consistency between the other files and the recommended data set.



\section{The plotting tools}
\label{autoplottingsystem}
A set of plotting tools for the database can be found in the directories names ``codepy3''. These tools are coded in python, and use the matplotlib library for plotting. The tools are designed to be used with the database, and are able to plot the cross sections.
A more detailed description of the tools, their use, and their implementation can be found in the README file in the codepy3 directory as well as the documentation file ``Atmociad\_doc\_v2.pdf''.

The vast majority of the images in the present document are automatically generated using these tools.

\section{The ASCII files}

Files in ASCII, as computed by Aeroplanets, are located in the directory ASCII. These are compiled from the recommended data, on regular basis (but the database in XML is likely more up-to-date).









%\include{appendix/xmlplotting}
%\include{appendix/webinterface}
\printnomenclature
\addcontentsline{toc}{chapter}{Nomenclature}


\bibliographystyle{plainnat}
\bibliography{atmociad}
\addcontentsline{toc}{chapter}{Bibliography}


\end{document}
