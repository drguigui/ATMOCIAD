
\clearpage
\section{Cross section of  impact with  Protons}
\subsection{Inelastic Cross Sections}

%\subsubsection{Legend for the properties}
%\paragraph{ R }: Recommended cross section for the processus. It is used in the main file. The selection of the recommended cross section is based on the quality of the data (e.g. errorbars, comparison with other experiments), the possibility of extrapolation, and the origin of the work, coupled with the consistency (sum of recommended cross sections ~ Total cross section)
%\paragraph{ U }: Estimated uncertainty: sometimes, the uncertainty is not given, because of theoretical work... The authors of the database have to estimate the uncertainty, but the quality of that estimation can be questionable. Moreover, when data from different sources have been adapted (e.g. for extrapolation), the uncertainty can be modified...
%\paragraph{ E }: Validated for extrapolation: the extrapolation of these cross sections is plausible. For example, when an analytic function has been applied...



\input{../proton/resultat/CO2.tex}
\input{../proton/resultat/CO.tex}
\input{../proton/resultat/H2.tex}
\input{../proton/resultat/H.tex}
\input{../proton/resultat/N2.tex}
\input{../proton/resultat/O2.tex}
\input{../proton/resultat/O.tex}

\clearpage

\section{Cross section of  impact with  Hydrogen}


\input{../hydrogen/resultat/CO2.tex}
\input{../hydrogen/resultat/N2.tex}
\input{../hydrogen/resultat/O2.tex}
\input{../hydrogen/resultat/O.tex}
